\section{Conclusions}
In this paper, we have recalled the previous works done on angular multigrid
method. For one dimensional geometry, this scheme has been proven to be very 
efficient compared to DSA. When using Fokker-Planck cross sections, the
spectral radius of the method is bounded by 0.6 while SI+DSA is bounded by 1. 
For multidimensional problems, the generalized scheme does
not show the same behavior. The angular multigrid method needs to be
stabilized
by a filter which degrades the spectral radius. The spectral radius is inferior 
to the one of DSA but it can still be arbitrary close to one. 
Unlike the previous methods which solved the $S_n$ equations using the Source 
Iteration method, we recast the angular multigrid method as a preconditioner 
for a Krylov solver. The method being stabilized with a Krylov solver, two 
versions were tested. One uses the sequence $S_n,S_{\lceil\frac{n}{2}\rceil},
\hdots,S_4,P1SA$, while the other uses the sequence $S_n,
S_{\lceil\frac{n}{2}\rceil},\hdots,S_2,DSA$.
The first sequence was proposed in \cite{multigrid_1d}, the second in 
\cite{multigrid_2d}. In \cite{multigrid_2d}, the authors did not use the first
sequence because it is known to be unstable for multidimensional problem when 
used with SI. Both sequences were tried using GMRES as Krylov solver and the 
second one was shown to be more 
efficient, i.e. the number of GMRES iterations is smaller for highly 
anisotropic medium and P1SA (PD) is harder to invert than
DSA (SPD). The number of GMRES iterations and the time needed
to solve the equations were compared for sweep preconditioning, DSA
preconditioning and angular multigrid-DSA preconditioning. The angular
multigrid was shown to be always faster and needed less iterations than DSA.
For highly anisotropic scattering, GMRES with angular multigrid
preconditioning is much faster than GMRES with DSA or sweep preconditioning.
