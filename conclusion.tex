\section{Conclusion}
In this paper, we recall the previous works done on the topic. The angular
multigrid method in one dimension has been proven to be very efficient
compared to DSA. For multidimensional problems, the generalized scheme does
not show the same advantages. The spectral radius is inferior to the one of
DSA but it can still be arbitrary close to one. Unlike the previous methods
which solved the $S_n$ equations using the Source Iteration method, we
recast the angular multigrid method as a preconditioner for a Krylov solver.
The method being stabilized with a Krylov solver, two versions were possible :
\begin{itemize}
\item using the sequence $S_n,S_{\lceil\frac{n}{2}\rceil},\hdots,S_4,P1SA$ 
\item using the sequence $S_n,S_{\lceil\frac{n}{2}\rceil},\hdots,S_2,DSA$
\end{itemize}
The first sequence was the one proposed in \cite{multigrid_1d}, while the
second was proposed in \cite{multigrid_2d}. The authors in
\cite{multigrid_2d}, did not use the first sequence because it is known to be
unstable when used with SI. Both sequence where tried in this paper and the
second one was shown to be more efficient i.e. the number of iterations is
larger for highly anisotropic medium and P1SA (PD) is harder to invert than
DSA (SPD). The number of iterations of the Krylov solver and the time needed
to solve the equation was compared for Sweep preconditioning, DSA
preconditioning and angular multigrid DSA preconditioning. The angular
multigrid was shown to be always faster and needed less iterations than DSA.
When the anisotropy is high enough the angular multigrid method is by far the
fastest while the anisotropy is not high enough the extra-work to solve it
makes it less efficient than the sweep preconditioning.
