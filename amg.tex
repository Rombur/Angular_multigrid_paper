\section{Angular multigrid}
\subsection{Equation} 
The equation that we want to solve is the Boltzmann equation :
\begin{equation}
\bo \cdot \bn \Psi(\br,\bo,E) + \Sigma_t(\br,\bo,E) \Psi(\br,\bo,E) =
\int_{4\pi}\!\!\int_{0}^{\infty}\Sigma_s(\br,\bo\cdot\bo',E'\rightarrow
E)\Psi(\br,\bo',E')\ dE'd\bo'+Q(\br,\bo,E) 
\label{transport}
\end{equation}
where :
\begin{itemize}
\item $\Psi$ is the angular flux.
\item $\Sigma_t$ is the total macroscopic cross section.
\item $\Sigma_s$ is the macroscopic differential scattering cross section.
\item $Q$ is a volumetric source.
\item $\br$ is the position.
\item $E$ is the energy.
\item $\bo = (\mu,\phi)$.
\item $\mu$ is the cosine pf the directional polar angle.
\item $\phi$ is the directional azimuthal angle.
\end{itemize}
Now, we discretize this equation in energy using the standard multigroup
method \cite{reuss} and in angle using the discrete ordinate method
\cite{reuss} :
\begin{equation}
\bo_d \cdot \bn \Psi_d^g(\br) + \Sigma_{t,g}^g(\br) \Psi_d^g(\br) = \sum_{g'=0}^G
\sum_{l=0}^L\sum_{m=-l}^l \frac{2l+1}{4\pi}\Sigma_{s,l}^{g'\rightarrow
g}\Phi_{l,m}^{g'}(\br)Y_l^m(\bo_d) + Q_d^g(\br)
\label{discretized}
\end{equation}
where :
\begin{itemize}
\item $\Psi_d^g(\br)=\int_{E^{g}}^{E^{g-1}}\Psi(\br,\bo_d,E)\ dE$.
\item $\Sigma_{t,d}^g(\br) = \frac{\int_{E^{g}}^{E^{g-1}}\Sigma_t(\br,\bo_d,E)
\Psi(\br,\bo_d,E) dE}{\int_{E^g}^{E^{g-1}}\Psi(\br,\bo_d,E) dE}$.
\item $\Sigma_{s,l}$ is the Legendre moment of degree $l$ of $\Sigma_s$.
\item $Q_d^g(\br) = \int_{E^g}^{E^{g-1}}Q(\br,\bo_d,E) dE$.
\item $\bo_d = (\mu_d,\phi_d)$.
\item $Y_l^m$ are the spherical harmonics.
\item $\Phi_{l,m}(\br,E)= \int_{4\pi}\Psi(\br,\bo,E)Y_l^{m,*}(\bo)\ d\bo$.
\end{itemize}
This equation still needs to be discretized in space. We used linear
discontinuous finite elements but other methods could be used. Equation
(\ref{discretized}) can be written as :
\begin{equation}
L \bs{\Psi} = M\Sigma D \bs{\Psi} + \bs{Q}
\label{operator}
\end{equation}
where :
\begin{itemize}
\item $L = \bo\cdot \bullet + \Sigma_t \bullet$
\item $\bs{\Psi}$ is a vector containing all the $\Psi_d$.
\item $\bs{Q}$ is a vector containing all the $Q_d$.
\item $\Sigma$ is a diagonal matrix with the element $\Sigma_{s,l}$ as
diagonal elements.
\item $M$ is the moment-to-discrete matrix ($\bs{\Psi}=M\bs{\Phi}$).
\item $D$ is the discrete-to-moment matrix ($\bs{\Phi}=D\bs{\Psi}$).
\end{itemize}
Equation (\ref{operator}) can be solved using the Source Iteration method or
Krylov Solver.\\
The Source Iteration (SI) method can be written as follows :
\begin{equation}
L\bs{\Psi}^k = M\Sigma D \bs{\Psi}^{k-1} + \bs{Q}
\end{equation}
The spectral radius of SI can become arbitrary close to one in diffusive
medium. The most common method to accelerate the convergence is to use the
Diffusion Synthetic Acceleration \cite{adams}. The SI+DSA scheme is given by :
\begin{align}
&\bs{\Phi}^{k+1/2} = DL^{-1}M\Sigma\bs{\Phi}^k + DL^{-1}Q\\
&\delta\bs{\Phi}^k = A \(\bs{\Phi}^{k+1/2}-\bs{\Phi}^k\)\\
&\bs{\Phi}^{k+1} = \bs{\Phi^{k+1/2}} + \delta \bs{\Phi}^k
\end{align}
where $A$ is the DSA operator.\\
When using a Krylov solver equation (\ref{operator}) can be rewritten as :
\begin{equation}
(I-DL^{-1}M\Sigma) \bs{\Phi} = DL^{-1}\bs{Q}
\label{krylov}
\end{equation}
where $I$ is the identity matrix. DSA can also help to speed up the
convergence of the Krylov solver. First we need to rewrite the accelerated
scheme :
\begin{align}
&\bs{\Phi}^{k+1} = \bs{\Phi}^{k}+\delta\bs{\Phi}^{k}\\
&\bs{\Phi}^{k+1} = DL^{-1}M\Sigma\bs{\Phi}^k+DL^{-1}\bs{Q}+A(\bs{\Phi}^{k+1/2}
-\bs{\Phi}^k)\\
&\bs{\Phi}^{k+1} = DL^{-1}M\Sigma\bs{\Phi}^k+DL^{-1}\bs{Q}+A(DL^{-1}M\Sigma
\bs{\Phi}^k+DL^{-1}\bs{Q}-\bs{\Phi}^{k})\\
&\bs{\Phi}^{k+1} = \((I+A)DL^{-1}M\Sigma-A\)\bs{\Phi}^k + (I+A)DL^{-1} Q
\end{align}
Thus, the equation that we need to solve with the Krylov solver is :
\begin{equation}
\((I+A)(I-DL^{-1}M\Sigma)\)\bs{\Phi} = (I+A)DL^{-1}Q
\end{equation}       
\subsection{Previous work}
DSA uses a diffusion equation to speed up the convergence. Because of this,
only the scalar flux and the current can be accelerated. This is a problem if
the scattering is highly anisotropic like for charged particles transport and
therefore others method have to be used \cite{kassem,multigrid_1d}. Morel and
Manteuffel have proposed an angular multigrid method to accelerate the SI
calculation of the one dimension within-group $S_n$ equations when the scattering 
is highly anisotropic \cite{multigrid_1d}. If we define :
\begin{equation}
Half(n) = \left\{
\begin{aligned}
&\frac{n}{2}, &\textrm{if $\frac{n}{2}$ is even}\\
&\frac{n}{2}+1, &\textrm{if $\frac{n}{2}$ is odd}
\end{aligned}
\right.
\end{equation}
The scheme works as follows : 
\begin{enumerate}
\item Perform a transport sweep for the $S_n$ equations.
\item Perform a transport sweep for the $S_{n_2}$ equations with a $P_{n_2-1}$
expansions using the $S_n$ residual as the inhomogeneous source, where
$n_2=Half(n)$.
\item Continue coarsening the angular grid by a factor 2 (i.e., according to
the definition of ``$Half$'') until a sweep has been performed for the $S_4$
equations.
\item Solve the diffusion equation with a $P_1$ expansion for the $S_4$
residual as the inhomogeneous source.
\item Add the Legendre moments of the diffusion solution to the Legendre
moments of the $S_4$ iterate to obtain the accelerated $S_4$ iterate.
\item Continue to add the corrections from each coarse grid to the finer grid
above to obtain the accelerated $S_n$ moments.
\end{enumerate}
Every time a transport sweep is performed, the optimal transport correction
needs to be used \cite{multigrid_1d}. For a $P_{n-1}$ expansion of the cross
sections, the corrected cross sections are given by :
\begin{equation}
\Sigma^* = \Sigma -\frac{\Sigma_{s,n/2}+\Sigma_{s,n-1}}{2}
\end{equation}
This correction is optimal because it minimizes the ``high-frequency'' angular
errors and because for an infinite homogeneous medium, the smoothing factor is
given by :
\begin{equation}
\rho_s =
\max(|\Sigma_{s,n/2}|/\Sigma_{s,0},|\Sigma_{s,n/2+1}|/\Sigma_{s,0},\hdots,
|\Sigma_{s,n-1}|/\Sigma_{s,0})
\end{equation}
One way to approximate the electron scattering cross section is to use the
Fokker-Planck scattering sections :
\begin{equation}
\Sigma_{s,l} = \frac{\alpha}{2} (L(L+1)-l(l+1))\ \ \ l=0,\hdots,L
\end{equation}
where $\alpha$ is the momentum transfer \cite{morel_81}. In one dimension,
DSA/P1SA become less efficient as $\Sigma_{s,l}$ $(l>0)$ becomes closer to
$\Sigma_{s,0}$ with increasing anisotropy order L. In the limit as
$L\rightarrow \infty$ becomes closer to $\Sigma_{s,0}$ with increasing
anisotropy order $L$. In the limit as $L\rightarrow \infty$, DSA/P1SA no
longer accelerate the convergence (the spectral radius tends to 1.0 and thus,
the acceleration scheme is inefficient).  However, the spectral radius of the
angular multigrid method has an upper bound of 0.6 when $L$ goes to infinity.
In the multidimensional case, DSA/P1SA become unstable when both the zeroth
and the first flux moments are accelerated and $\frac{\Sigma_{s,1}}{\Sigma_t}
\geq 0.5$ \cite{multisweep}. In \cite{multigrid_2d}, the authors modified the
one dimensional angular multigrid method by accelerating only the zeroth flux
moments with the DSA but to compensate for the loss of a more effective DSA,
the lowest transport sweep is an $S_2$ sweep instead of an $S_4$. Moreover, a
filter is now needed to stabilize the method. Therefore, the angular multigrid 
method was modified as follows \cite{multigrid_2d} :
\begin{enumerate}
\item Perform a transport sweep for the $S_n$ equations.
\item Perform a transport sweep for the $S_{n_2}$ equations with a $P_{n_2}$
for 2-D problem and a $P_{n_2}+1$ for 3-D problem expansion for the $S_n$
residual as the inhomogeneous source, where $n_2=Half(n)$.
\item Continue coarsening the angular grid by a factor (i.e., according to the
definition of ``$Half$'') until a sweep has been performed for the $S_2$
equations.
\item Solve the diffusion equation with a $P_0$ expansion for the $S_2$
residual as the inhomogeneous source. 
\item Apply a diffusive filter to the corrections from step 2 and 3. Without a
filter, the method is unstable.
\item Add the corrections from steps 4 and 5 to the Legendre moments of the
$S_n$ iterate to obtain the accelerated $S_n$ moments.
\end{enumerate}
The filter stabilizes the method which otherwise would diverge but it reduces
the efficiency of the angular multigrid and the spectral radius can become
close to one when $n$, and, therefore, $L$ becomes large.
\subsection{Angular multigrid with Krylov solver}
In this paper, we propose to abandon the SI method and to use the angular
multigrid as a preconditioner for the Krylov solver. The successive sweeps are
now different stages of a preconditioner. Using a method similar to the one
we used to write the equation for the preconditioned Krylov solver, we get
\hbox{successively :}
\begin{align}
& \Phi_n^{(l+1/2)} = D_n L_n^{-1} M_n \Sigma_n \Phi_n^{(l)} + D_n L_n^{-1} Q\\
& \delta \Phi_{n/2}^{(l)} = D_{n/2} L_{n/2}^{-1} M_{n/2} \Sigma_{n/2}
R_{n\rightarrow n/2} \(\Phi_n^{(l+1/2)}-\Phi_n^{(l)}\)\\
& \hdots\\
& \delta \Phi_2^{(l)} = D_2 L_2^{-1} M_2 \Sigma_2 R_{4\rightarrow 2} \delta \Phi_4\\
& \delta \Phi_0^{(l)} = \mathcal{T}_0^{-1} \delta \Phi_2^{(l)} \\
& \Phi_n^{(l+1)} = \Phi_n^{(l+1/2)} + P_{n/2 \rightarrow n} \delta
\Phi_{n/2}^{(l)} + \hdots + P_{2 \rightarrow n} \delta \Phi_{2}^{(l)} + P_{0
\rightarrow n} \delta \Phi_{0}^{(l)}
\end{align}

\begin{equation}
\begin{split}
\Phi_n^{(l+1)} =& T_n \Phi_n^{(l)} + D_n L_n^{-1} Q +
P_{n/2 \rightarrow n} \(T_{n/2}
R_{n\rightarrow n/2} \(\Phi_n^{(l+1/2)} - \Phi_n^{(l)}\)\)+\hdots \\
&+ P_{2 \rightarrow n} T_2 R_{4\rightarrow 2} \delta
\Phi_{4}  + P_{0\rightarrow n} \mathcal{T}_0^{-1} \delta \Phi_2^{(l)}\\
=& T_n \Phi_n^{(l)} + D_n L_n^{-1} Q + P_{n/2 \rightarrow
n} \(T_{n/2} R_{n \rightarrow n/2}\(T_n \Phi_n^{(l)} +D_n L_n^{-1} Q -\Phi_n^{(l)}
\)\)\\
& +\hdots + P_{2\rightarrow n} T_2 R_{4\rightarrow 2} 
\(T_4 R_{8\rightarrow 4}\( \hdots \(T_n \Phi_n^{(l)} + D_n L_n^{-1} Q -
 \Phi_n^{(l)}\)\) \) \\ 
&+ P_{0\rightarrow n} \mathcal{T}_0^{-1} R_{2\rightarrow 0}\(T_2 R_{4\rightarrow 2} 
\(\hdots\(T_n \Phi_n^{(l)}+D_n L_n^{-1}Q-\Phi_n^{(l)}\)\)\)\\
=& \(T_n + P_{n/2\rightarrow n} T_{n/2} R_{n\rightarrow n/2}\(T_n-I\)+
 \hdots + P_{2\rightarrow n} T_2 R_{4\rightarrow}
\(T_4 R_{8\rightarrow 4} \(\hdots\(T_n -I\)\)\)\right.\\ 
&\left. +P_{0\rightarrow n} \mathcal{T}_0 R_{2\rightarrow 0}  \(T_2
R_{4\rightarrow 2} (\hdots \(T_n-I\))\)\) \Phi_n^{(l)}
+\(I+P_{n/2\rightarrow n} T_{n/2} R_{n\rightarrow
n/2}+ \hdots + \right.\\
&\left. P_{2\rightarrow n} T_2 R_{4\rightarrow 2} \(T_4
 R_{8\rightarrow 4}\(\hdots
\(T_{n/2}R_{n\rightarrow n/2}\)\)\)+\right.\\
& \left. P_{0\rightarrow n} \mathcal{T}_0^{-1}R_{2\rightarrow 0}
\(T_2 R_{4\rightarrow 2}\(\hdots\(T_{n/2}R_{n\rightarrow n/2}\)\)\)\)
D_nL_n^{-1} Q
\end{split}
\end{equation}

\begin{equation}
\begin{split}
&(I+P_{\frac{n}{2}\rightarrow n }
T_{\frac{n}{2}} (I+P_{\frac{n}{4}\rightarrow\frac{n}{2}}T_{\frac{n}{4}} (\hdots
(I+P_{0\rightarrow 2} \mathcal{T}_0 R_{2\rightarrow 0})\hdots)
R_{\frac{n}{2}\rightarrow \frac{n}{4}})R_{n\rightarrow\frac{n}{2}})(I-T_n)
\bs{\Phi}_n =\\
& (I+P_{\frac{n}{2}\rightarrow n} T_{\frac{n}{2}} (I+P_{\frac{n}{4}
\rightarrow \frac{n}{2}} T_{\frac{n}{4}} (\hdots (I+P_{0 \rightarrow
2}\mathcal{T}_0 R_{2\rightarrow 0})\hdots)R_{\frac{n}{2}\rightarrow
\frac{n}{4}})R_{n\rightarrow \frac{n}{2}} ) D_n L_n^{-1} Q
\end{split}
\end{equation}  
where :
\begin{itemize}
\item $T_n = D_n L_n^{-1}M_n \Sigma_n$, the subscript $n$ is there to remind
that we are solving the $S_n$ equations.
\item $\mathcal{T}_0$ is the DSA operator.
\item $P_{\frac{n}{2}\rightarrow n}$ is the projection matrix of
$\bs{\Phi}_{\frac{n}{2}}$ to $\bs{\Phi}_n$.
\item $R_{n\rightarrow \frac{n}{2}}$ is the restriction matrix of
$\bs{\Phi}_{\frac{n}{2}}$ to $\bs{\Phi}_n$.
\end{itemize}
In this paper, we will use the Modified Interior Penalty (MIP) DSA \cite{mip}
that has been derived from the discretized transport equations using linear
discontinuous finite elements.
