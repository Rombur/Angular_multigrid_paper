\section{Angular multigrid}
In this section, we first give the multigroup $S_n$ transport equations. Then,
the Source Iteration technique is recalled and the $S_n$ equations are
recast to be solved by a Krylov solver. Next, the previous works on angular
multigrid methods are presented. Finally, this section is concluded by the derivation
of our method.  
\subsection{Multigroup $S_n$ transport equations} 
Charged particles transport can be described by the Boltzmann equation :
\begin{equation}
\bo \cdot \bn \Psi(\br,\bo,E) + \Sigma_t(\br,\bo,E) \Psi(\br,\bo,E) =
\int_{4\pi}\!\!\int_{0}^{\infty}\Sigma_s(\br,\bo\cdot\bo',E'\rightarrow
E)\Psi(\br,\bo',E')\ dE'd\bo'+Q(\br,\bo,E) 
\label{transport}
\end{equation}
where :
\begin{itemize}
\item $\Psi$ is the angular flux.
\item $\Sigma_t$ is the total macroscopic cross section.
\item $\Sigma_s$ is the macroscopic differential scattering cross section.
\item $Q$ is a volumetric source.
\item $\br$ is the position.
\item $E$ is the energy.
\item $\bo = (\mu,\phi)$.
\item $\mu$ is the cosine pf the directional polar angle.
\item $\phi$ is the directional azimuthal angle.
\end{itemize}
Standard boundary conditions can be applied to (\ref{transport}); the most
commonly employed is in incoming flux boundary conditions :
\begin{equation}
\Psi(\br,\bo,E) = g(\br,\bo,E)\textrm{ for }\bo\cdot\bs{n}<0\textrm{ and
}\br\in \partial D
\label{boundary}
\end{equation} 
where $\partial D$ is the boundary of the problem. If $g=0$, equation
(\ref{boundary}) yields the vacuum boundary conditions.
Now, we discretize equation (\ref{transport}) in energy using the standard multigroup
method \cite{reuss} and angle using the discrete ordinate method
\cite{reuss} :
\begin{equation}
\bo_d \cdot \bn \Psi_d^g(\br) + \Sigma_{t,g}^g(\br) \Psi_d^g(\br) = \sum_{g'=0}^G
\sum_{l=0}^L\sum_{m=-l}^l \frac{2l+1}{4\pi}\Sigma_{s,l}^{g'\rightarrow
g}\Phi_{l,m}^{g'}(\br)Y_l^m(\bo_d) + Q_d^g(\br)
\label{discretized}
\end{equation}
where :
\begin{itemize}
\item $\Psi_d^g(\br)=\int_{E^{g}}^{E^{g-1}}\Psi(\br,\bo_d,E)\ dE$.
\item $\Sigma_{t,d}^g(\br) = \frac{\int_{E^{g}}^{E^{g-1}}\Sigma_t(\br,\bo_d,E)
\Psi(\br,\bo_d,E) dE}{\int_{E^g}^{E^{g-1}}\Psi(\br,\bo_d,E) dE}$.
\item $\Sigma_{s,l}$ is the Legendre moment of degree $l$ of $\Sigma_s$.
\item $Q_d^g(\br) = \int_{E^g}^{E^{g-1}}Q(\br,\bo_d,E) dE$.
\item $\bo_d = (\mu_d,\phi_d)$.
\item $Y_l^m$ are the spherical harmonics.
\item $\Phi_{l,m}(\br,E)= \int_{4\pi}\Psi(\br,\bo,E)Y_l^{m,*}(\bo)\ d\bo$.
\end{itemize}
This equation still needs to be discretized in space. We used linear
discontinuous finite elements but other discretizations could be used. Equation
(\ref{discretized}) can be written as :
\begin{equation}
L \bs{\Psi} = M\Sigma D \bs{\Psi} + \bs{Q}
\label{operator}
\end{equation}
where :
\begin{itemize}
\item $L = \bo\cdot \bullet + \Sigma_t \bullet$
\item $\bs{\Psi}$ is a vector containing all the $\Psi_d$.
\item $\bs{Q}$ is a vector containing all the $Q_d$.
\item $\Sigma$ is a diagonal matrix with the element $\Sigma_{s,l}$ as
diagonal elements.
\item $M$ is the moment-to-discrete matrix ($\bs{\Psi}=M\bs{\Phi}$).
\item $D$ is the discrete-to-moment matrix ($\bs{\Phi}=D\bs{\Psi}$).
\end{itemize}
Equation (\ref{operator}) can be solved using the Source Iteration method or
Krylov Solver.\\
The Source Iteration method at iteration $k$ can be written as follows :
\begin{equation}
L\bs{\Psi}^k = M\Sigma D \bs{\Psi}^{k-1} + \bs{Q}
\end{equation}
The spectral radius of SI can become arbitrary close to one in diffusive
medium. The most common method to accelerate the convergence is to use 
Diffusion Synthetic Acceleration \cite{adams}. The SI+DSA scheme is given by :
\begin{align}
&\bs{\Phi}^{k+1/2} = DL^{-1}M\Sigma\bs{\Phi}^k + DL^{-1}Q\\
&\delta\bs{\Phi}^k = \mathcal{T}_0 R_{n\rightarrow 0} 
\(\bs{\Phi}^{k+1/2}-\bs{\Phi}^k\)\\
&\bs{\Phi}^{k+1} = \bs{\Phi}^{k+1/2} + P_{0/1 \rightarrow n} \delta \bs{\Phi}^k
\end{align}
where $\mathcal{T}_0$ is the DSA operator, $R_{n\rightarrow 0}$ is the
restriction matrix of $\bs{\Phi}_{n}$ to $\bs{\Phi}_0$ and $P_{0/1 \rightarrow
n}$ is the projection matrix of $\bs{\Phi}_0$ or $\bs{\Phi}_1$, depending if
only the zeroth or the zeroth and the first moment are accelerated, to
$\bs{\Phi}_n$.\\
When using a Krylov solver equation (\ref{operator}) can be rewritten as :
\begin{equation}
(I-DL^{-1}M\Sigma) \bs{\Phi} = DL^{-1}\bs{Q}
\label{krylov}
\end{equation}
where $I$ is the identity matrix. Equation (\ref{krylov}) is equation
(\ref{operator}) preconditioned by $DL^{-1}$ (sweep preconditioning). DSA can also 
help to speed up the
convergence of the Krylov solver. First, the accelerated scheme needs to be
rewritten :
\begin{align}
&\bs{\Phi}^{k+1} = \bs{\Phi}^{k}+\delta\bs{\Phi}^{k}\\
&\bs{\Phi}^{k+1} = DL^{-1}M\Sigma\bs{\Phi}^k+DL^{-1}\bs{Q}+P_{0/1\rightarrow
n} \mathcal{T}_0 R_{n\rightarrow 0}(\bs{\Phi}^{k+1/2}
-\bs{\Phi}^k)\\
&\bs{\Phi}^{k+1} = DL^{-1}M\Sigma\bs{\Phi}^k+DL^{-1}\bs{Q}+P_{0/1\rightarrow
n} \mathcal{T}_0 R_{n\rightarrow 0} (DL^{-1}M\Sigma
\bs{\Phi}^k+DL^{-1}\bs{Q}-\bs{\Phi}^{k})\\
&\bs{\Phi}^{k+1} = \((I+P_{0/1\rightarrow n} \mathcal{T}_0 R_{n\rightarrow 0} )
DL^{-1}M\Sigma-A\)\bs{\Phi}^k + (I+P_{0/1\rightarrow n}\mathcal{T}_0
R_{n\rightarrow 0})DL^{-1} Q
\end{align}
Thus, the equation that we need to solve with the Krylov solver is :
\begin{equation}
\((I+P_{0/1 \rightarrow n}\mathcal{T}_0 R_{n\rightarrow 0})(I-DL^{-1}M\Sigma)\)
\bs{\Phi} = (I+P_{0/1 \rightarrow n}\mathcal{T}_0 R_{n\rightarrow 0})DL^{-1}Q
\end{equation}       
\subsection{Previous work}
Like mentioned previously, only the scalar flux and the current can be
accelerated with DSA. To accelerate higher moments, other methods have to be
used \cite{kassem,multigrid_1d}. Morel and Manteuffel have proposed an angular 
multigrid method to accelerate the inner SI calculation of the one dimension 
$S_n$ equations when the scattering is highly anisotropic 
\cite{multigrid_1d}. Their idea was to use a variation of the extended
transport correction \cite{lathrop} to attenuate the ``upper half'' of the
flux moments thanks to a transport sweep. The ``lower half'' of the flux
moments is accelerated using the $S_n$ equations with only half the angles.
These $S_n$ equations with half the angles are themselves accelerated using the same
technique. The number of angles is divided by 2 until the $S_4$ equations are
solved. At this point, a $P1$ equation is used to accelerate the $S_4$. If we define :
\begin{equation}
Half(n) = \left\{
\begin{aligned}
&\frac{n}{2}, &\textrm{if $\frac{n}{2}$ is even}\\
&\frac{n}{2}+1, &\textrm{if $\frac{n}{2}$ is odd}
\end{aligned}
\right.
\end{equation}
The scheme works as follows : 
\begin{enumerate}
\item Perform a transport sweep for the $S_n$ equations.
\item Perform a transport sweep for the $S_{n_2}$ equations with a $P_{n_2-1}$
expansions using the $S_n$ residual as the inhomogeneous source, where
$n_2=Half(n)$.
\item Continue coarsening the angular grid by a factor 2 (i.e., according to
the definition of ``$Half$'') until a sweep has been performed for the $S_4$
equations.
\item Solve the $P_1 equations$ with a $P_1$ expansion for the $S_4$
residual as the inhomogeneous source.
\item Add the Legendre moments of the diffusion solution to the Legendre
moments of the $S_4$ iterate to obtain the accelerated $S_4$ iterate.
\item Continue to add the corrections from each coarse grid to the finer grid
above to obtain the accelerated $S_n$ moments.
\end{enumerate}
Every time a transport sweep is performed, the optimal transport correction
needs to be used \cite{multigrid_1d}. For a $P_{n-1}$ expansion of the cross
sections, the corrected cross sections are given by :
\begin{equation}
\Sigma_{j}^* = \Sigma_{j} -\frac{\Sigma_{s,n/2}+\Sigma_{s,n-1}}{2}\ 
\textrm{ with }j=t \textrm{ or }s,l
\end{equation}
This correction is optimal because for an infinite homogeneous medium, it minimizes 
the ``high-frequency'' angular errors since the smoothing factor is given by :
\begin{equation}
\rho_s =
\max(|\Sigma_{s,n/2}|/\Sigma_{s,0},|\Sigma_{s,n/2+1}|/\Sigma_{s,0},\hdots,
|\Sigma_{s,n-1}|/\Sigma_{s,0})
\end{equation}
To compare the effectiveness of the angular multigrid method with DSA, we can
use Fokker-Planck scattering cross section. This cross section approximate 
the electron scattering cross section :
\begin{equation}
\Sigma_{s,l} = \frac{\alpha}{2} (L(L+1)-l(l+1))\ \ \ l=0,\hdots,L
\end{equation}
where $\alpha$ is the momentum transfer or transport corrected scattering
cross section \cite{multigrid_1d,morel_81}. In one dimension,
DSA becomes less efficient as $\Sigma_{s,l}$ $(l>0)$ becomes closer to
$\Sigma_{s,0}$ with increasing anisotropy order L. In the limit as
$L\rightarrow \infty$ becomes closer to $\Sigma_{s,0}$ with increasing
anisotropy order $L$. In the limit as $L\rightarrow \infty$, DSA no
longer accelerate the convergence (the spectral radius tends to 1.0 and thus,
the acceleration scheme is inefficient). However, the spectral radius of the
angular multigrid method has an upper bound of 0.6 when $L$ goes to infinity.
In the multidimensional case, DSA become unstable when both the zeroth
and the first flux moments are accelerated and $\frac{\Sigma_{s,1}}{\Sigma_t}
\geq 0.5$ \cite{multisweep}. In \cite{multigrid_2d}, the authors modified the
one dimensional angular multigrid method by accelerating only the zeroth flux
moments with the DSA but to compensate for the loss of a more effective DSA,
the lowest transport sweep is an $S_2$ sweep instead of an $S_4$. Moreover, a
filter is now needed to stabilize the method. Therefore, the angular multigrid 
method was modified as follows \cite{multigrid_2d} :
\begin{enumerate}
\item Perform a transport sweep for the $S_n$ equations.
\item Perform a transport sweep for the $S_{n_2}$ equations with a $P_{n_2}$
for 2-D problem and a $P_{n_2}+1$ for 3-D problem expansion for the $S_n$
residual as the inhomogeneous source, where $n_2=Half(n)$.
\item Continue coarsening the angular grid by a factor (i.e., according to the
definition of ``$Half$'') until a sweep has been performed for the $S_2$
equations.
\item Solve the diffusion equation with a $P_0$ expansion for the $S_2$
residual as the inhomogeneous source. 
\item Apply a diffusive filter to the corrections from step 2 and 3. Without a
filter, the method is unstable.
\item Add the corrections from steps 4 and 5 to the Legendre moments of the
$S_n$ iterate to obtain the accelerated $S_n$ moments.
\end{enumerate}
The filter stabilizes the method which otherwise would diverge but it reduces
the efficiency of the angular multigrid and the spectral radius can become
close to one when $L$ becomes large.
\subsection{Angular multigrid with Krylov solver}
In this paper, we propose to abandon the SI method and to use the angular
multigrid as a preconditioner for the Krylov solver. The successive sweeps are
now different stages of a preconditioner. Since, a Krylov solver is used to
stabilize the method, two variations are possible :
\begin{itemize}
\item the coarsest level can be DSA.
\item the coarsest level can be P1SA.
\end{itemize}
First, we will show the angular multigrid using DSA and then, the angular
multigrid using P1SA.
\subsubsection{DSA}
Using a method similar to the one we used to write the equation for the 
preconditioned Krylov solver, we get
\hbox{successively :}
\begin{align}
& \Phi_n^{(k+1/2)} = D_n L_n^{-1} M_n \Sigma_n \Phi_n^{(k)} + D_n L_n^{-1} Q\\
& \delta \Phi_{n/2}^{(k)} = D_{n/2} L_{n/2}^{-1} M_{n/2} \Sigma_{n/2}
R_{n\rightarrow n/2} \(\Phi_n^{(k+1/2)}-\Phi_n^{(k)}\)\\
& \hdots\\
& \delta \Phi_2^{(k)} = D_2 L_2^{-1} M_2 \Sigma_2 R_{4\rightarrow 2} \delta \Phi_4\\
& \delta \Phi_0^{(k)} = \mathcal{T}_0^{-1} R_{2\rightarrow 0} \delta \Phi_2^{(k)} \\
& \Phi_n^{(k+1)} = \Phi_n^{(k+1/2)} + P_{n/2 \rightarrow n} \delta
\Phi_{n/2}^{(k)} + \hdots + P_{2 \rightarrow n} \delta \Phi_{2}^{(k)} + P_{0
\rightarrow n} \delta \Phi_{0}^{(k)}
\end{align}

\begin{equation}
\begin{split}
\Phi_n^{(k+1)} =& T_n \Phi_n^{(k)} + D_n L_n^{-1} Q +
P_{n/2 \rightarrow n} \(T_{n/2}
R_{n\rightarrow n/2} \(\Phi_n^{(k+1/2)} - \Phi_n^{(k)}\)\)+\hdots \\
&+ P_{2 \rightarrow n} T_2 R_{4\rightarrow 2} \delta
\Phi_{4}  + P_{0\rightarrow n} \mathcal{T}_0^{-1} R_{2\rightarrow 0} \delta 
\Phi_2^{(k)}\\
=& T_n \Phi_n^{(k)} + D_n L_n^{-1} Q + P_{n/2 \rightarrow
n} \(T_{n/2} R_{n \rightarrow n/2}\(T_n \Phi_n^{(k)} +D_n L_n^{-1} Q -\Phi_n^{(k)}
\)\)\\
& +\hdots + P_{2\rightarrow n} T_2 R_{4\rightarrow 2} 
\(T_4 R_{8\rightarrow 4}\( \hdots \(T_n \Phi_n^{(k)} + D_n L_n^{-1} Q -
 \Phi_n^{(k)}\)\) \) \\ 
&+ P_{0\rightarrow n} \mathcal{T}_0^{-1} R_{2\rightarrow 0}\(T_2 R_{4\rightarrow 2} 
\(\hdots\(T_n \Phi_n^{(k)}+D_n L_n^{-1}Q-\Phi_n^{(k)}\)\)\)\\
=& \(T_n + P_{n/2\rightarrow n} T_{n/2} R_{n\rightarrow n/2}\(T_n-I\)+
 \hdots + P_{2\rightarrow n} T_2 R_{4\rightarrow}
\(T_4 R_{8\rightarrow 4} \(\hdots\(T_n -I\)\)\)\right.\\ 
&\left. +P_{0\rightarrow n} \mathcal{T}_0 R_{2\rightarrow 0}  \(T_2
R_{4\rightarrow 2} (\hdots \(T_n-I\))\)\) \Phi_n^{(k)}
+\(I+P_{n/2\rightarrow n} T_{n/2} R_{n\rightarrow
n/2}+ \hdots + \right.\\
&\left. P_{2\rightarrow n} T_2 R_{4\rightarrow 2} \(T_4
 R_{8\rightarrow 4}\(\hdots
\(T_{n/2}R_{n\rightarrow n/2}\)\)\)+\right.\\
& \left. P_{0\rightarrow n} \mathcal{T}_0^{-1}R_{2\rightarrow 0}
\(T_2 R_{4\rightarrow 2}\(\hdots\(T_{n/2}R_{n\rightarrow n/2}\)\)\)\)
D_nL_n^{-1} Q
\end{split}
\end{equation}
where $T_n = D_n L_n^{-1}M_n \Sigma_n$, the subscript $n$ is there to remind
that we are solving the $S_n$ equations.
\begin{equation}
\begin{split}
&(I+P_{{n}/{2}\rightarrow n }
T_{{n}/{2}} (I+P_{{n}/{4}\rightarrow {n}/{2}}T_{{n}/{4}} (\hdots
(I+P_{0\rightarrow 2} \mathcal{T}_0 R_{2\rightarrow 0})\hdots)
R_{{n}/{2}\rightarrow {n}/{4}})R_{n\rightarrow{n}/{2}})(I-T_n)
\bs{\Phi}_n =\\
& (I+P_{{n}/{2}\rightarrow n} T_{{n}/{2}} (I+P_{{n}/{4}
\rightarrow {n}/{2}} T_{{n}/{4}} (\hdots (I+P_{0 \rightarrow
2}\mathcal{T}_0 R_{2\rightarrow 0})\hdots)R_{{n}/{2}\rightarrow
{n}/{4}})R_{n\rightarrow {n}/{2}} ) D_n L_n^{-1} Q
\end{split}
\end{equation}  
In this paper, we will use the Modified Interior Penalty (MIP) DSA \cite{mip}. 
This DSA uses bilinear discontinuous finite elements. We will recall here the
weak form of this DSA :
\begin{equation}
b(\phi,\phi^*) = l(\phi^*)
\end{equation}
with :
\begin{equation}
\begin{split}
b(\phi,\phi^*) =& (\Sigma_a \phi,\phi^*)_{\partial D} +
\(D\bn\phi,\bn\phi^*\)_{\mathcal{D}} + \(\kappa_e\llb\phi\rrb,\llb\phi^*\rrb\)_{E_h^i}
+ \(\llb\phi\rrb,\ldb D\partial_n \phi\rdb\)_{E_h^i} +\\
&(\ldb D \partial_n \phi\rdb,\llb\phi^*\rrb)_{E_h^i} + (\kappa_e\phi,\phi^*)_{\partial
D^d}-\frac{1}{2} \(\phi,D\partial D \partial_n \phi^*\)_{\partial
\mathcal{D}^d} - \frac{1}{2} (D \partial_n \phi,\phi^*)_{\partial \mathcal{D}^d}
\end{split}
\end{equation}
\begin{equation}
l(\phi^*) = (Q_0,\phi^*)_{\mathcal{D}} + (J^{inc},\phi^*)_{\partial
\mathcal{D}^r}
\end{equation}
where :
\begin{itemize}
\item $(f,g)_K = \int_K fg\ d\br$
\item $\la f,g\ra_e = \int_e |\bo_m\cdot \bs{n}_e| fg\ ds$
\item $\(f,g\)_{\mathcal{D}} = \sum_{K\in\mathbb{T}_h} (f,g)_K$
\item $\la f,g\ra_{E_h^i} = \sum_{e\in E_h^i} \la f,g\ra_e$ 
\item $\mathbb{T}_h$ is the mesh used to discretize the domain $\mathcal{D}$
into nonoverlapping elements $K$, $E_h^i$ is the set of interior edges,
$\mathcal{D}$ is the spatial domain, $\partial \mathcal{D}^d$ is the boundary
of the domain with Dirichlet condition and $\partial \mathcal{D}^r$ is the
boundary of the domain with reflective condition.
\item $\Sigma_a$ is the absorption macroscopic cross section.
\item $D$ is the diffusion coefficient.
\item $\partial_n = \bs{n}\cdot \bn$ where $\bs{n}$ is the outward unit
normal.
\item $\llb \phi\rrb = \phi^{+}-\phi^{-}$ is the jump of at the interface
between two elements.
\item $\ldb\phi\rdb = \frac{\phi^++\phi^-}{2}$ is the mean of $\phi$ at the
interface between two elements.
\item $\phi^{\pm}=\lim_{s\to 0^{\pm}}$ is the mean of $\phi$ at the interface
between two elements.
\item $\kappa_e = \max\(\kappa_e^{IP},\frac{1}{4}\)$ with
$\kappa_e^{IP}=\left\{
\begin{aligned}
&\frac{c(p^+)}{2}\frac{D^+}{h_{\bot}^+} + \frac{c(p^-)}{2}
\frac{D^-}{h_{\bot}^-} &\textrm{ on interior edges, i.e., }e\in E_h^i\\
&c(p)\frac{D}{h_{\bot}} & \textrm{ on boundary edges,
i.e.,i }e\in\partial\mathcal{D}^d
\end{aligned}
\right.$\\
$c(p)$ is given by $c(p)=2p(p+1)$, $p$ is the polynomial order and $h_{\bot}$
is the length of the cell in the direction orthogonal to the edge $e$.
\end{itemize}

\subsubsection{P1SA}
We use the P1SA defined as P1C in \cite{yaqi} :
\begin{equation}
\begin{split}
& (I+P_{n/2\rightarrow n} T_{n/2} (I+P_{n/4\rightarrow n/2}
T_{n/4}(\hdots(I+P_{1\rightarrow 4}\mathcal{T}_1 R_{4\rightarrow
1})\hdots)R_{n/2 \rightarrow n/4})R_{n\rightarrow n/2}) (I-T_n)\bs{\Phi}_n = \\
& (I+P_{n/2\rightarrow n} T_{n/2} (I+P_{n/4\rightarrow n/2}
T_{n/4}(\hdots(I+P_{1\rightarrow 4}\mathcal{T}_1 R_{4\rightarrow
1})\hdots)R_{n/2\rightarrow n/4})R_{n\rightarrow n/2}) D_n L_n^{-1} Q
\end{split}
\end{equation}
where $\mathcal{T}_1$ is the P1SA operator.
\begin{equation}
b_{P1C}(\Phi,\bs{J},\Phi^*,\bs{J}^*) = l_{P1C}(\Phi^*,\bs{J}^*)
\end{equation}
with :
\begin{equation}
\begin{split}
b_{P1}(\Phi,\bs{J},\Phi^*,\bs{J}^*) = & (\Sigma_a \Phi,\Phi^*)_{\mathcal{D}} +
(3\Sigma_{tr} \bs{J},\bs{J}^*)_{\mathcal{D}} + (\bn
\Phi,\bs{J}^*)_{\mathcal{D}} - (\bs{J},\bn \Phi^*)_{\mathcal{D}}\\
&+\frac{1}{4} \(\llb\Phi\rrb,\llb\Phi^*\rrb\)_{E_h^i} +
\(\llb\Phi\rrb,\ldb\bs{J}\cdot\bs{n}\rdb\)_{E_h^i} - (\ldb
\bs{J}\cdot\bs{n}\rdb, \llb\Phi^*\rrb)_{E_h^i}\\
&+\frac{9}{16}\(\llb\bs{J}\cdot\bs{n}\rrb,\llb\bs{J}^*\cdot\bs{n}\rrb\)_{E_h^i}
+ \frac{9}{16}\(\llb\bs{J}\rrb,\llb\bs{J}^*\rrb\)_{E_h^i}\\
&+\frac{1}{4}(\Phi,\Phi^*)_{\partial \mathcal{D}^d} +
\frac{1}{2}(\Phi,\bs{J}^*\cdot\bs{n})_{\partial \mathcal{D}^d} - \frac{1}{2}
(\bs{J}\cdot\bs{n},\Phi^*)_{\partial\mathcal{D}^d}\\
&+\frac{9}{16}(\bs{J},\bs{J}^*)_{\partial
\mathcal{D}^d}+\frac{9}{16}(\bs{J}\cdot\bs{n},\bs{J}^*\cdot\bs{n})_{\partial 
\mathcal{D}^d} + \frac{9}{4} (\bs{J}\cdot\bs{n},\bs{J}^*\cdot\bs{n})_{\partial
\mathcal{D}^r}
\end{split}
\end{equation}
\begin{equation}
l(\Phi^*,\bs{J}^*) = (Q_0,\Phi^*)_{\mathcal{D}} +
(3\bs{Q}_1,\bs{J}^*)_{\mathcal{D}}
\end{equation}
where :
\begin{itemize}
\item $\bs{J}$ is the current or first moment of the flux
\item $\Sigma_{tr}=\Sigma_t-\Sigma_{s,1}$
\item $\bs{n}$ is the unit normal associated to an edge. The direction is
chosen arbitrary except for the edges on the boundary. For this ones, $\bs{n}$
is arbitrary.\\
This scheme is positive definite but non-symmetric.
\end{itemize}
