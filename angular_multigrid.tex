\documentclass{article}
\usepackage{amsmath}
\usepackage{array}
\usepackage{color}
\usepackage{graphicx}
\usepackage{float} % utiliser H pour forcer a mettre l'image ou on veut
\usepackage{lscape} % utilisation du mode paysage
\usepackage{mathbbol} % permet d'avoir le vrai symbol pour les reels grace a mathbb
\usepackage{enumerate} % permet d'utiliser enumerate
\usepackage{marvosym} % permet d'avoir le symbol pour le nucleaire
\usepackage{moreverb} % permet d'utiliser verbatimtab : conservation la tabulation
\usepackage{stmaryrd} % permet d'utiliser \llbrackedt et \rrbracket : double crochet


\setlength {\textwidth}{16cm}
\setlength {\textheight}{21cm}
\setlength {\oddsidemargin}{0cm}
\setlength{\headsep}{5pt} 

\newcommand\bn{\boldsymbol{\nabla}}
\newcommand\bo{\boldsymbol{\Omega}}
\newcommand\br{\mathbf{r}}
\newcommand\la{\left\langle}
\newcommand\ra{\right\rangle}
\newcommand\bs{\boldsymbol}
\newcommand\red{\textcolor{red}}
\newcommand\ldb{\{\!\!\{}
\newcommand\rdb{\}\!\!\}}
\newcommand\llb{\llbracket}
\newcommand\rrb{\rrbracket}

\renewcommand{\(}{\left(}
\renewcommand{\)}{\right)}
\renewcommand{\[}{\left[}
\renewcommand{\]}{\right]}


\begin{document}
\title{An Angular Multigrid Acceleration Method for $S_n$ Equations with
Highly Forward-Peaked Scattering}
\author{} 
\date{}
\maketitle

%\begin{abstract}
Due to the inefficiency of the Diffusion Synthetic Acceleration (DSA) scheme for
highly forward-peaked scattering, 1-D and 2-D angular multigrid methods have been developed. 
The 1-D scheme is unconditionally effective and very efficient but the 2-D
method can become inefficient when the anisotropy 
increases too much. This is due to the fact that the scheme needs a filter to be
stable. In this work, the angular multigrid method has been recast as a
preconditioner for a Krylov solver instead of being used as an acceleration
method for a Source Iteration solver. The Krylov solver stabilizes the method
which does not need a filter anymore. The new method is shown to be faster and
to need significantly fewer GMRES iterations than DSA preconditioning and sweep
preconditioning on two test problems.
\end{abstract}


\section{Introduction}
Charged-particle transport applications are ubiquitous and include radiotherapy, 
electronics hardening, etc. Charged particles interact through Coulomb
interactions. Therefore, each particle undergoes a very large number of
interactions; each slightly changing the energy and the particle.
Because of this, the scattering is highly forward-peaked. Thus, the magnitude 
of the angular moments of the particle flux distribution decreases very slowly 
and, a high number of moments is necessary to solve the problem accurately.\\
If the scattering ratio $\(c=\frac{\Sigma_{s,0}}{\Sigma_t}\)$ is high, it can be difficult to solve the transport
equation. To speed up the convergence, acceleration schemes such as Diffusion
Synthetic Acceleration (DSA) must be used. DSA schemes focus on accelerating
the weakly anisotropic 
part of the flux, only the zeroth or the zeroth and the first flux 
moments are accelerated \cite{adams}. This is very efficient for neutron transport 
but not for charged-particle transport \cite{multigrid_1d}. The spectral radius of the Source
Iteration (SI) method accelerated by DSA becomes arbitrary close to one when the 
anisotropy increases and the scattering ratio is close to one. Therefore, other methods have been 
developed to accelerate higher order flux moments 
\cite{multigrid_1d,multigrid_2d}.\\
Morel and Manteuffel developed in \cite{multigrid_1d} an angular multigrid
method for the $S_n$ equations in one dimensional geometry. Their angular
multigrid method uses a sequence of lower order transport operator and a
$P_1$ operator to accelerate higher moments. For the scheme to be able to
work, an ``optimal'' transport correction has to be applied to each transport
operator. This transport correction minimizes the ``high-frequency'' angular
errors. The results were very encouraging and Pautz et al. generalized this method 
to multidimensional geometries \cite{multigrid_2d}. This multidimensional
extension of the original angular multigrid method needs to be stabilized
because the scheme amplifies the high-frequency error modes
instead of damping them. They proposed a 
filter which allows the scheme to be always stable but which also reduces the 
efficiency of the method compared to the one dimensional case. However, the spectral 
radius can now become arbitrary close to one when the anisotropy increases.
The angular multigrid methods proposed so far were only applied in the context
of SI but the system of linear equations can also be tackled using non-stationary 
Krylov solvers, such as GMRES. In \cite{ttg}, the authors applied GMRES to
the single group $S_n$ equations with isotropic scattering and 
summarizes the advantageous features of GMRES as \hbox{follows :} ``using DSA as a 
preconditioner for GMRES(m) removes the consistency requirement that plagues 
DSA-accelerated source iteration in multidimensional problems.'' Driven by this
property, we propose to use the angular multigrid as a preconditioner for a 
Krylov solver instead of an acceleration scheme for the Source Iteration
method with application to problems with highly forward-peaked scattering. 
The Krylov solver does not require the method to have a spectral
radius below one to be effective and, therefore, the filter used in
\cite{multigrid_2d} becomes
unnecessary. A code solving the $S_n$ equations using SI with DSA can easily
be modified to use preconditioned Krylov solver. The paper is organized as 
follows; in Section 2, we present the
notation and the discretized transport equation and we recall the
previous works done on this topic and the method that we propose. In Section
3, we show some results and in Section 4, we end with some conclusions.


\section{Review of iterative solution techniques applied to the $S_n$
transport equation}
In this section, we first give the multigroup $S_n$ transport equations. Next,
the Source Iteration technique is recalled and the $S_n$ equations are
recast to be solved by a Krylov solver. Previous work on angular
multigrid methods applied as preconditioner to SI is presented. Finally, we
conclude this section with the derivation of the new method.  
\subsection{Multigroup $S_n$ equations} 
Charged particles transport can be described by the Boltzmann equation
\cite{graal} which discretized in energy using the standard multigroup method
\cite{reuss} and in angle using the discrete ordinate method \cite{reuss}, is
given by :
\begin{equation}
\bo_d \cdot \bn \Psi_d^g(\br) + \Sigma_{t,g}^g(\br) \Psi_d^g(\br) = \sum_{g'=0}^G
\sum_{l=0}^L\sum_{m=-l}^l \frac{2l+1}{4\pi}\Sigma_{s,l}^{g'\rightarrow
g}\Phi_{l,m}^{g'}(\br)Y_l^m(\bo_d) + Q_d^g(\br)
\label{discretized}
\end{equation}
where $\br$ is the position in domain $\mathcal{D}$, $E$ is the energy,
$\bo_d=(\mu_d,\varphi_d)$, $\mu_d$ is the cosine of the directional polar angle,
$\varphi_d$ is the directional azimuthal angle, 
$\Psi_d^g(\br)=\int_{E^{g}}^{E^{g-1}}\Psi(\br,\bo_d,E)\ dE$, $\Psi$ is
the angular flux, $Q_d^g(\br) = \int_{E^g}^{E^{g-1}}Q(\br,\bo_d,E) dE$, $Q$ is
a volumetric source, $\Sigma_{t}^g(\br)$ is the multigroup total 
macroscopic cross section, $\Sigma_{s,l}$ is the Legendre moment of 
degree $l$ of $\Sigma_s$ and $\Phi_{l,m}(\br,E)= \int_{4\pi}\Psi(\br,\bo,E)Y_l^{m,*}(\bo)\ d\bo
\approx \sum_{d=1} w_d \Psi(\br,\bo_d,E)Y_l^{m,*}(\bo_d)$ where $Y_l^m$ 
are the spherical harmonics.
Standard boundary conditions can be applied to (\ref{discretized}); the most
commonly employed is the incoming flux boundary condition :
\begin{equation}
\Psi(\br,\bo,E) = g(\br,\bo,E)\textrm{ for }\bo\cdot\bs{n}<0\textrm{ and
}\br\in \partial \mathcal{D},
\label{boundary}
\end{equation} 
where $\partial \mathcal{D}$ is the boundary of the domain $\mathcal{D}$. If $g=0$, equation
(\ref{boundary}) yields vacuum boundary conditions.\\
This equation still needs to be discretized in space but the method proposed
in this paper is not restricted to a particular spatial discretization. For
the results presented here, we use bilinear discontinuous finite elements in
two dimensional space problem \cite{dgfem}.\\ 
Equation (\ref{discretized}) can be written using operator notation as :
\begin{equation}
L \bs{\Psi} = M\Sigma D \bs{\Psi} + \bs{Q}
\label{operator}
\end{equation}
where $\bs{\Psi}$ is a vector containing all the $\Psi_d$, $\bs{Q}$ is a
vector containing all the $Q_d$, $\Sigma$ is a diagonal matrix with the
element $\Sigma_{s,l}$ as diagonal elements, $M$ is the moment-to-discrete
matrix $(\bs{\Psi} = M\bs{\Phi})$, $D$ is the discrete-to-moments matrix
$(\Phi=D\Psi)$ (the moment entries are dictated  by the dimensionality of the
problem) and $L = \bo\cdot + \Sigma_t$ is the streaming + total interaction operator,
inverted direction by direction (transport sweeps). The inversion can be done
locally for each mesh cell for each discrete directional fluxes in the
quadrature.\\
The cross section used here are the Fokker-Planck cross sections :
\begin{equation}
\Sigma_{s,l}=\frac{\alpha}{2} (L(L+1)-l(l+1))\ \ l=0,\hdots,L
\end{equation}
where $\alpha$ is the restricted momentum transfer \cite{morel_96}.
These cross sections approximate the electron scattering cross sections
\cite{morel_81} and require Galerkin quadratures , i.e.,
$D=M^{-1}$ \cite{graal}, to obtain accurate results.
\subsection{Solution techniques}
Equation (\ref{operator}) can be solved using the Source Iteration method or a
Krylov method. The Source Iteration method at the $k^{th}$ iteration is given
by :
\begin{equation}
\bs{\Phi}^{(k+1)} = DL^{-1}M\Sigma \bs{\Phi}^{(k)} + DL^{-1}\bs{Q}
\end{equation}
The spectral radius of SI can become arbitrary close to one when the
scattering ratio $\(\frac{\Sigma_{s,0}}{\Sigma_t}\)$ is close to one. 
To accelerate the convergence of SI, the Diffusion Synthetic
Acceleration scheme \cite{adams} is commonly used. The SI+DSA scheme is given by :
\begin{align}
&\bs{\Phi}^{(k+1/2)} = DL^{-1}M\Sigma\bs{\Phi}^{(k)} + DL^{-1}\bs{Q}\\
&\delta\bs{\Phi}^{(k)} = \mathcal{T}_0^{-1} R_{n\rightarrow 0} 
\(\bs{\Phi}^{(k+1/2)}-\bs{\Phi}^{(k)}\)\\
&\bs{\Phi}^{(k+1)} = \bs{\Phi}^{(k+1/2)} + P_{0/1 \rightarrow n} \delta
\bs{\Phi}^{(k)}
\end{align}
Finally, we get :
\begin{equation}
\bs{\Phi}^{(k+1)} = \((I+P_{0/1\rightarrow n} \mathcal{T}_0^{-1} R_{n\rightarrow 0} )
DL^{-1}M\Sigma-P_{0/1\rightarrow n}\mathcal{T}_0^{-1} R_{n\rightarrow 0}\)
\bs{\Phi}^{(k)} + (I+P_{0/1\rightarrow n}\mathcal{T}_0^{-1}
R_{n\rightarrow 0})DL^{-1} Q
\end{equation}
where $\mathcal{T}_0$ is the DSA operator, $R_{n\rightarrow 0}$ is the
restriction matrix of $\bs{\Phi}_{n}$ to $\bs{\Phi}_0$ and $P_{0/1 \rightarrow
n}$ is the projection matrix of $\bs{\Phi}_0$ or $\bs{\Phi}_1$, depending
whether only the zeroth or the zeroth and the first moment are accelerated, to
$\bs{\Phi}_n$.\\
When using a Krylov solver, equation (\ref{operator}) has to be rewritten as :
\begin{equation}
(I-DL^{-1}M\Sigma) \bs{\Phi} = DL^{-1}\bs{Q}
\label{krylov}
\end{equation}
where $I$ is the identity matrix. Equation (\ref{krylov}) is equation
(\ref{operator}) preconditioned by $DL^{-1}$ (sweep preconditioning). DSA can also 
help to speed up the convergence of the Krylov solver. The system of equations 
solved with the Krylov method is :
\begin{equation}
\((I+P_{0/1 \rightarrow n}\mathcal{T}_0^{-1} R_{n\rightarrow 0})(I-DL^{-1}M\Sigma)\)
\bs{\Phi} = (I+P_{0/1 \rightarrow n}\mathcal{T}_0^{-1} R_{n\rightarrow 0})DL^{-1}Q
\end{equation}       
\subsection{Review of previous angular multigrid work}
\subsubsection{One dimensional geometry}
As mentioned previously, only the zeroth and the first flux moments can be
accelerated with DSA. To accelerate higher moments, other methods have to be
used. Morel and Manteuffel have proposed an angular 
multigrid method to accelerate the inner SI calculation of the one dimension 
$S_n$ equations when the scattering is highly anisotropic 
\cite{multigrid_1d}. The motivation to use a variation of the extended
transport correction \cite{lathrop} to attenuate the ``upper half'' of the
flux moments (higher frequencies) thanks to transport sweeps. The ``lower half'' 
of the flux moments (lower frequencies) is accelerated using the $S_{n/2}$ 
equations. These $S_{n/2}$ equations are themselves 
accelerated using $S_{n/4}$ equations. The order of the transport operator is 
divided by two 
until the $S_4$ equations are solved. At this point, the $P_1$ equations are 
used to accelerate the $S_4$ equations. We define :
\begin{equation}
Half(n) = \left\{
\begin{aligned}
&\frac{n}{2}, &\textrm{if $\frac{n}{2}$ is even}\\
&\frac{n}{2}+1, &\textrm{if $\frac{n}{2}$ is odd}
\end{aligned}
\right.
\end{equation}
Their scheme works as follows : 
\begin{enumerate}
\item Perform a transport sweep for the $S_n$ equations.
\item Perform a transport sweep for the $S_{n_2}$ equations with a $P_{n_2-1}$
expansions using the $S_n$ residual as the inhomogeneous source, where
$n_2=Half(n)$.
\item Continue coarsening the angular grid by a factor two (i.e., according to
the definition of ``$Half$'') until a sweep has been performed for the $S_4$
equations.
\item Solve the $P_1$ equations (P1SA) with a $P_1$ expansion of the $S_4$
residual as the inhomogeneous source.
\item Add the Legendre moments of the diffusion solution to the Legendre
moments of the $S_4$ iterate to obtain the accelerated $S_4$ iterate.
\item Continue to add the corrections from each coarse grid to the finer grid
above to obtain the accelerated $S_n$ moments.
\end{enumerate}
Every time a transport sweep is performed, the optimal transport correction
needs to be used \cite{multigrid_1d}. For a $P_{n-1}$ expansion of the cross
sections, the corrected cross sections are given by :
\begin{equation}
\Sigma_{j}^* = \Sigma_{j} -\frac{\Sigma_{s,n/2}+\Sigma_{s,n-1}}{2}\ 
\textrm{ with }j=t \textrm{ or }s,l
\end{equation}
This correction is said to be optimal because for an infinite homogeneous medium, 
it minimizes the ``high-frequency'' angular errors, the smoothing factor being 
given by :
\begin{equation}
\rho_s =
\max(|\Sigma_{s,n/2}|/\Sigma_{s,0},|\Sigma_{s,n/2+1}|/\Sigma_{s,0},\hdots,
|\Sigma_{s,n-1}|/\Sigma_{s,0})
\end{equation}
To compare the effectiveness of the angular multigrid method with DSA, 
Fokker-Planck scattering cross sections can be used. In one dimension,
DSA becomes less efficient as $\Sigma_{s,l}$ $(l>0)$ becomes closer to
$\Sigma_{s,0}$. Therefore, in the limit as $L\rightarrow \infty$, DSA no
longer accelerates the convergence of SI when Fokker-Planck cross section are used. 
The spectral radius tends to 1.0 and thus,
the acceleration scheme is inefficient. However, the spectral radius of the
angular multigrid method has an upper bound of 0.6 when $L$ goes to
infinity.
\subsubsection{Extension to multidimensional geometries}
In the multidimensional case, DSA becomes unstable when both the zeroth
and the first flux moments are accelerated and $\frac{\Sigma_{s,1}}{\Sigma_t}
\geq 0.5$ \cite{multisweep}. In \cite{multigrid_2d}, the authors modified the
one dimensional angular multigrid method by accelerating only the zeroth flux
moment with the DSA and by using $S_2$ as lowest transport sweep
instead of an $S_4$. Moreover, a
filter is now needed to stabilize the method. Therefore, the angular multigrid 
method was modified as follows \cite{multigrid_2d} :
\begin{enumerate}
\item Perform a transport sweep for the $S_n$ equations.
\item Perform a transport sweep for the $S_{n_2}$ equations with a $P_{n_2}$
for 2-D problem and a $P_{n_2+1}$ for 3-D problem expansion for the $S_n$
residual as the inhomogeneous source, where $n_2=Half(n)$.
\item Continue coarsening the angular grid by a factor two (i.e., according to the
definition of ``$Half$'') until a sweep has been performed for the $S_2$
equations.
\item Solve the diffusion equation with a $P_0$ expansion for the $S_2$
residual as the inhomogeneous source. 
\item Apply a diffusive filter to the corrections from steps 2 and 3 (without
this filter, the method is unstable).
\item Add the corrections from steps 4 and 5 to the Legendre moments of the
$S_n$ iterate to obtain the accelerated $S_n$ moments.
\end{enumerate}
The filter stabilizes the method which otherwise would diverge but it reduces
the efficiency of the angular multigrid. The spectral radius can become
arbitrary close to one when $L$ becomes large \cite{multigrid_2d}.
\section{Angular multigrid with Krylov solver}
In this paper, we propose to abandon SI as the solver for the $S_n$ equations 
and to use the angular multigrid method as a preconditioner for the Krylov solver. 
The successive corrections of the angular multigrid acceleration form
now different stages of a preconditioner used int the Krylov solver. Since, a Krylov solver is used to
stabilize the method, two variations are possible :
\begin{itemize}
\item the coarsest level can be DSA (ANMG-DSA).
\item the coarsest level can be P1SA (ANMG-P1SA).
\end{itemize}
First, we present the angular multigrid using DSA and then, the angular
multigrid using P1SA. Later, these two versions are be compared.
\subsubsection{ANMG-DSA}
Using a method similar to the one we used to write the equation for the 
preconditioned Krylov solver, we obtain \hbox{successively :}
\begin{align}
& \Phi_n^{(k+1/2)} = D_n L_n^{-1} M_n \Sigma_n \Phi_n^{(k)} + D_n L_n^{-1} Q\\
& \delta \Phi_{n/2}^{(k)} = D_{n/2} L_{n/2}^{-1} M_{n/2} \Sigma_{n/2}
R_{n\rightarrow n/2} \(\Phi_n^{(k+1/2)}-\Phi_n^{(k)}\)\\
& \hdots\\
& \delta \Phi_2^{(k)} = D_2 L_2^{-1} M_2 \Sigma_2 R_{4\rightarrow 2} \delta \Phi_4\\
& \delta \Phi_0^{(k)} = \mathcal{T}_0^{-1} R_{2\rightarrow 0} \delta \Phi_2^{(k)} \\
& \Phi_n^{(k+1)} = \Phi_n^{(k+1/2)} + P_{n/2 \rightarrow n} \delta
\Phi_{n/2}^{(k)} + \hdots + P_{2 \rightarrow n} \delta \Phi_{2}^{(k)} + P_{0
\rightarrow n} \delta \Phi_{0}^{(k)}
\end{align}

\begin{equation}
\begin{split}
\Phi_n^{(k+1)} =& T_n \Phi_n^{(k)} + D_n L_n^{-1} Q +
P_{n/2 \rightarrow n} \(T_{n/2}
R_{n\rightarrow n/2} \(\Phi_n^{(k+1/2)} - \Phi_n^{(k)}\)\)+\hdots \\
&+ P_{2 \rightarrow n} T_2 R_{4\rightarrow 2} \delta
\Phi_{4}^{(k)} + P_{0\rightarrow n} \mathcal{T}_0^{-1} R_{2\rightarrow 0} \delta 
\Phi_2^{(k)}\\
=& T_n \Phi_n^{(k)} + D_n L_n^{-1} Q + P_{n/2 \rightarrow
n} \(T_{n/2} R_{n \rightarrow n/2}\(T_n \Phi_n^{(k)} +D_n L_n^{-1} Q -\Phi_n^{(k)}
\)\)\\
& +\hdots + P_{2\rightarrow n} T_2 R_{4\rightarrow 2} 
\(T_4 R_{8\rightarrow 4}\( \hdots \(T_n \Phi_n^{(k)} + D_n L_n^{-1} Q -
 \Phi_n^{(k)}\)\) \) \\ 
&+ P_{0\rightarrow n} \mathcal{T}_0^{-1} R_{2\rightarrow 0}\(T_2 R_{4\rightarrow 2} 
\(\hdots\(T_n \Phi_n^{(k)}+D_n L_n^{-1}Q-\Phi_n^{(k)}\)\)\)\\
=& \(T_n + P_{n/2\rightarrow n} T_{n/2} R_{n\rightarrow n/2}\(T_n-I\)+
 \hdots + P_{2\rightarrow n} T_2 R_{4\rightarrow}
\(T_4 R_{8\rightarrow 4} \(\hdots\(T_n -I\)\)\)\right.\\ 
&\left. +P_{0\rightarrow n} \mathcal{T}_0^{-1} R_{2\rightarrow 0}  \(T_2
R_{4\rightarrow 2} (\hdots \(T_n-I\))\)\) \Phi_n^{(k)}
+\(I+P_{n/2\rightarrow n} T_{n/2} R_{n\rightarrow
n/2}+ \hdots + \right.\\
&\left. P_{2\rightarrow n} T_2 R_{4\rightarrow 2} \(T_4
 R_{8\rightarrow 4}\(\hdots
\(T_{n/2}R_{n\rightarrow n/2}\)\)\)+\right.\\
& \left. P_{0\rightarrow n} \mathcal{T}_0^{-1}R_{2\rightarrow 0}
\(T_2 R_{4\rightarrow 2}\(\hdots\(T_{n/2}R_{n\rightarrow n/2}\)\)\)\)
D_nL_n^{-1} Q
\end{split}
\end{equation}
where $T_n = D_n L_n^{-1}M_n \Sigma_n$. We use the subscript $n$ as a reminder
that we are solving the $S_n$ equations.\\
Finally, the system of equations which has to be solved is :
\begin{equation}
\begin{split}
&(I+P_{{n}/{2}\rightarrow n }
T_{{n}/{2}} (I+P_{{n}/{4}\rightarrow {n}/{2}}T_{{n}/{4}} (\hdots
(I+P_{0\rightarrow 2} \mathcal{T}_0^{-1} R_{2\rightarrow 0})\hdots)
R_{{n}/{2}\rightarrow {n}/{4}})R_{n\rightarrow{n}/{2}})(I-T_n)
\bs{\Phi}_n =\\
& (I+P_{{n}/{2}\rightarrow n} T_{{n}/{2}} (I+P_{{n}/{4}
\rightarrow {n}/{2}} T_{{n}/{4}} (\hdots (I+P_{0 \rightarrow
2}\mathcal{T}_0^{-1} R_{2\rightarrow 0})\hdots)R_{{n}/{2}\rightarrow
{n}/{4}})R_{n\rightarrow {n}/{2}} ) D_n L_n^{-1} Q
\end{split}
\label{anmg-dsa}
\end{equation}  
At this point, it is necessary to choose a DSA scheme. Among different
schemes (see \cite{adams}), the Modified Interior Penalty (MIP) DSA was
chosen \cite{mip}. This DSA is based on a second-order discontinuous of a
diffusion equation, as opposed to a mixed of $P1$ formulation.
It uses bilinear discontinuous finite elements, which is
the same spatial discretization that the one chosen for the
$S_n$ equations and it has been shown to be always stable for isotropic
scattering and is symmetric positive definite (SPD), which makes it easy to invert.
We will recall here the weak form of this DSA :
\begin{equation}
b_{MIP}(\phi,\phi^*) = l_{MIP}(\phi^*)
\end{equation}
with :
\begin{equation}
\begin{split}
b_{MIP}(\phi,\phi^*) =& (\Sigma_a \phi,\phi^*)_{\partial D} +
\(\mathrm{D}\bn\phi,\bn\phi^*\)_{\mathcal{D}} + \(\kappa_e\llb\phi\rrb,
\llb\phi^*\rrb\)_{E_h^i}
+ \(\llb\phi\rrb,\ldb \mathrm{D}\partial_n \phi^*\rdb\)_{E_h^i} +\\
&(\ldb \mathrm{D} \partial_n \phi\rdb,\llb\phi^*\rrb)_{E_h^i} + 
(\kappa_e\phi,\phi^*)_{\partial
D^d}-\frac{1}{2} \(\phi,\mathrm{D}\partial_n \phi^*\)_{\partial
\mathcal{D}^d} - \frac{1}{2} (\mathrm{D} \partial_n \phi,\phi^*)_{\partial 
\mathcal{D}^d}
\end{split}
\end{equation}
\begin{equation}
l_{MIP}(\phi^*) = (Q_0,\phi^*)_{\mathcal{D}} 
\end{equation}
where :
\begin{itemize}
\item $(f,g)_{\mathcal{D}} = \sum_{K\in \mathbb{T}_h} \int_K fg\ d\br$ and 
$(f,g)_{E_h^i} = \sum_{e\in E_h^i} \int_e fg\ ds$
\item $\mathbb{T}_h$ is the mesh used to discretize the domain $\mathcal{D}$
into nonoverlapping elements $K$, $E_h^i$ is the set of interior edges,
$\mathcal{D}$ is the spatial domain, $\partial \mathcal{D}^d$ is the boundary
of the domain with Dirichlet condition and $\partial \mathcal{D}^r$ is the
boundary of the domain with reflective condition
\item $\Sigma_a$ is the absorption macroscopic cross section
\item $\mathrm{D}$ is the diffusion coefficient
\item $\partial_n = \bs{n}\cdot \bn$ where $\bs{n}$ is the outward unit
normal
\item $\llb \phi\rrb = \phi^{+}-\phi^{-}$ is the jump of at the interface
between two elements
\item $\ldb\phi\rdb = \frac{\phi^++\phi^-}{2}$ is the mean of $\phi$ at the
interface between two elements
\item $\phi^{\pm}(\br)=\lim_{s\to 0^{\pm}}\phi(\br+s\bs{n}_e)$, $\bs{n}_e$ is
the normal unit vector associated with a given edge $e$
\item $\kappa_e = \max\(\kappa_e^{IP},\frac{1}{4}\)$ with
$\kappa_e^{IP}=\left\{
\begin{aligned}
&\frac{c(p^+)}{2}\frac{D^+}{h_{\bot}^+} + \frac{c(p^-)}{2}
\frac{D^-}{h_{\bot}^-} &\textrm{ on interior edges, i.e. }e\in E_h^i\\
&c(p)\frac{D}{h_{\bot}} & \textrm{ on boundary edges,
i.e. }e\in\partial\mathcal{D}^d
\end{aligned}
\right.$\\
$c(p)$ is given by $c(p)=2p(p+1)$, $p$ is the polynomial order and $h_{\bot}$
is the length of the cell in the direction orthogonal to the edge $e$
\end{itemize}

\subsubsection{ANMG-P1SA}
Using P1SA and $S_4$ as the lowest $S_n$ order instead of DSA and $S_2$ in 
equation (\ref{anmg-dsa}) gives us :
\begin{equation}
\begin{split}
& (I+P_{n/2\rightarrow n} T_{n/2} (I+P_{n/4\rightarrow n/2}
T_{n/4}(\hdots(I+P_{1\rightarrow 4}\mathcal{T}_1^{-1} R_{4\rightarrow
1})\hdots)R_{n/2 \rightarrow n/4})R_{n\rightarrow n/2}) (I-T_n)\bs{\Phi}_n = \\
& (I+P_{n/2\rightarrow n} T_{n/2} (I+P_{n/4\rightarrow n/2}
T_{n/4}(\hdots(I+P_{1\rightarrow 4}\mathcal{T}_1^{-1} R_{4\rightarrow
1})\hdots)R_{n/2\rightarrow n/4})R_{n\rightarrow n/2}) D_n L_n^{-1} Q
\end{split}
\end{equation}
where $\mathcal{T}_1$ is the P1SA operator.\\
The P1SA used is the P1C defined in \cite{yaqi}. This scheme,
which is positive definite but non-symmetric, is given by :
\begin{equation}
b_{P1C}(\Phi,\bs{J},\Phi^*,\bs{J}^*) = l_{P1C}(\Phi^*,\bs{J}^*)
\end{equation}
with :
\begin{equation}
\begin{split}
b_{P1}(\Phi,\bs{J},\Phi^*,\bs{J}^*) = & (\Sigma_a \Phi,\Phi^*)_{\mathcal{D}} +
(3\Sigma_{tr} \bs{J},\bs{J}^*)_{\mathcal{D}} + (\bn
\Phi,\bs{J}^*)_{\mathcal{D}} - (\bs{J},\bn \Phi^*)_{\mathcal{D}}\\
&+\frac{1}{4} \(\llb\Phi\rrb,\llb\Phi^*\rrb\)_{E_h^i} +
\(\llb\Phi\rrb,\ldb\bs{J}\cdot\bs{n}_e\rdb\)_{E_h^i} - (\ldb
\bs{J}\cdot\bs{n}_e\rdb, \llb\Phi^*\rrb)_{E_h^i}\\
&+\frac{9}{16}\(\llb\bs{J}\cdot\bs{n}_e\rrb,\llb\bs{J}^*\cdot\bs{n}_e\rrb\)_{E_h^i}
+ \frac{9}{16}\(\llb\bs{J}\rrb,\llb\bs{J}^*\rrb\)_{E_h^i}\\
&+\frac{1}{4}(\Phi,\Phi^*)_{\partial \mathcal{D}^d} +
\frac{1}{2}(\Phi,\bs{J}^*\cdot\bs{n}_e)_{\partial \mathcal{D}^d} - \frac{1}{2}
(\bs{J}\cdot\bs{n}_e,\Phi^*)_{\partial\mathcal{D}^d}\\
&+\frac{9}{16}(\bs{J},\bs{J}^*)_{\partial
\mathcal{D}^d}+\frac{9}{16}(\bs{J}\cdot\bs{n}_e,\bs{J}^*\cdot\bs{n}_e)_{\partial 
\mathcal{D}^d} + \frac{9}{4} (\bs{J}\cdot\bs{n}_e,\bs{J}^*\cdot\bs{n}_e)_{\partial
\mathcal{D}^r}
\end{split}
\end{equation}
\begin{equation}
l(\Phi^*,\bs{J}^*) = (Q_0,\Phi^*)_{\mathcal{D}} +
(3\bs{Q}_1,\bs{J}^*)_{\mathcal{D}}
\end{equation}
where $\bs{J}$ is the current or first moment of the flux and 
$\Sigma_{tr}=\Sigma_t-\Sigma_{s,1}$.


\section{Results}
We use Fokker-Planck cross section a homogeneous domain by $5cm$ by $5cm$ and
50 by 50 cells. We use Fokker-Planck cross section
$\(\Sigma_{s,l}=\frac{\alpha}{2}\(L(L+1)-l(l+1)\)\)$ with $\alpha=1$
\cite{morel_81} and a Galerkin-Legendre-Chebyshev quadrature. There is an 
uniform isotropic source of intensity 10 $n/(cm^3s)$. We compare the number of 
GMRES iterations (the relative tolerance is set at $10^{-4}$) and the elapsed 
time for :
\begin{itemize}
\item Sweep preconditioning (S).
\item DSA preconditioning (DSA).
\item Angular multigrid preconditioning (MG).
\end{itemize}

\begin{table}[H]
\begin{center}
\begin{tabular}{|c|c|c|c|c|c|c|c|c|}
\hline
\multicolumn{3}{|c|}{$S_4$} & \multicolumn{3}{c|}{$S_r84$} & 
\multicolumn{3}{c|}{$S_{16}$} \\
\hline  
S & DSA & MG & S & DSA & MG & S & DSA & MG\\
\hline
50 & 28 & 17 & 215 & 67 & 23 & 992 & 175 & 42 \\
\hline
\end{tabular}
\caption{GMRES iterations}
\end{center}
\end{table}

\begin{table}[H]
\begin{center}
\begin{tabular}{|c|c|c|c|c|c|c|c|c|}
\hline
\multicolumn{3}{|c|}{$S_4$} & \multicolumn{3}{c|}{$S_8$} & 
\multicolumn{3}{|c|}{$S_{16}$} \\
\hline  
S & DSA & MG & S & DSA & MG & S & DSA & MG\\
\hline
269 & 1067 & 667 & 3380 & 4147 & 1331 & 68061 & 17595 & 5226 \\
\hline
\end{tabular}
\caption{Elapsed time (s)}
\end{center}
\end{table}



\section{Conclusions}
In this paper, we have recalled the previous works done on angular multigrid
method. For one dimensional geometry, this scheme has been proven to be very 
efficient compared to DSA. When using Fokker-Planck cross sections, the
spectral radius of the method is bounded by 0.6 while SI+DSA is bounded by 1. 
For multidimensional problems, the generalized scheme does
not show the same behavior. The angular multigrid method needs to be
stabilized
by a filter which degrades the spectral radius. The spectral radius is inferior 
to the one of DSA but it can still be arbitrary close to one. 
Unlike the previous methods which solved the $S_n$ equations using the Source 
Iteration method, we recast the angular multigrid method as a preconditioner 
for a Krylov solver. The method being stabilized with a Krylov solver, two 
versions were tested. One uses the sequence $S_n,S_{\lceil\frac{n}{2}\rceil},
\hdots,S_4,P1SA$, while the other uses the sequence $S_n,
S_{\lceil\frac{n}{2}\rceil},\hdots,S_2,DSA$.
The first sequence was proposed in \cite{multigrid_1d}, the second in 
\cite{multigrid_2d}. In \cite{multigrid_2d}, the authors did not use the first
sequence because it is known to be unstable for multidimensional problem when 
used with SI. Both sequences were tried using GMRES as Krylov solver and the 
second one was shown to be more 
efficient, i.e. the number of GMRES iterations is smaller for highly 
anisotropic medium and P1SA (PD) is harder to invert than
DSA (SPD). The number of GMRES iterations and the time needed
to solve the equations were compared for sweep preconditioning, DSA
preconditioning and angular multigrid-DSA preconditioning. The angular
multigrid was shown to be always faster and needed less iterations than DSA.
For highly anisotropic scattering, GMRES with angular multigrid
preconditioning is much faster than GMRES with DSA or sweep preconditioning.


\bibliographystyle{unsrt}
\bibliography{biblio}

\end{document}

