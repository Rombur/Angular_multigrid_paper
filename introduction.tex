\section{Introduction}
Highly anisotropic scattering is very common because it describes charged
particles scattering. The applications of charged particles transport are
legion and include radiotherapy, electronics hardening, etc. These particles 
interact through Coulomb interactions and therefore, undergo a lot of interactions 
but each one of them changes very slightly the energy and the direction of the 
particle. Because of this, the magnitude of the scattering moments decreases 
very slowly and a high number of moments is necessary to solve the problem 
accurately.\\
If the problem is very diffusive, it can be difficult to solve the transport
equation discretized by the discrete ordinate method. To speed up the
convergence, an acceleration scheme must be used. Standard acceleration schemes 
like Diffusion Synthetic Acceleration (DSA) focus on accelerating the isotropic 
part of the flux, only the zeroth or the zeroth and the first flux 
moments are accelerated \cite{adams}. This is very efficient for neutron transport 
but not for charged particle transport. The spectral radius of the Source
Iteration (SI) method accelerated by DSA becomes arbitrary close to one when the 
anisotropy increases in a diffusive medium. Therefore, other methods have been 
developed to accelerate higher order flux moments 
\cite{kassem,multigrid_1d,multigrid_2d}.\\
Morel an Manteuffel developed in \cite{multigrid_1d} an angular multigrid
method for the $S_n$ equations in one dimensional geometry. The angular
multigrid method uses a sequence of lower order transport operator and a
P1 operator to accelerate higher moments. For the scheme to be able to
work, an ``optimal'' transport correction has to be applied to each transport
operator. This transport correction minimizes the ``high-frequency'' angular
errors. The results were very encouraging and Pautz et al. generalized this method 
to multidimensional geometries \cite{multigrid_2d}. This multidimensional
angular multigrid method needs to be stabilized. They proposed a 
filter which allows the scheme to be always stable but reduces the 
efficiency of the method compare to the one dimensional case. The spectral 
radius can now become arbitrary close to one when the anisotropy increases.
These two schemes used the angular multigrid method as an acceleration scheme
for SI but this is not the only method to solve the $S_n$ equations. The system 
of linear equations can also be tackled using non-stationary Krylov solvers, 
such as GMRES. In \cite{ttg}, the authors 
summarizes the advantageous features of GMRES as follows : ``using DSA as a 
preconditioner for GMRES(m) removes the consistency requirement that plagues 
DSA-accelerated source iteration in multidimensional problems.'' Driven by this
property, we propose to use the angular multigrid as a preconditioner for a 
Krylov solver instead of an acceleration scheme for the Source Iteration
method. The Krylov solver stabilizes the method and therefore, the filter becomes
unnecessary. A code solving the $S_n$ equations using SI with DSA can easily
be modified to use preconditioned Krylov solver. The paper is organized as 
follows : in Section 2, we present the
notation and the discretized transport equation,  we recall the
previous works done on this topic and the method that we propose, in Section
3, we show some results and in Section 4, we end with the conclusion.
