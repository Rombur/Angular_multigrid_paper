\section{Introduction}
Charged-particle transport applications are ubiquitous and include radiotherapy, 
electronics hardening, etc. Charged particles interact through Coulomb
interactions. Therefore, each particle undergoes a very large number of
interactions; each slightly changing the energy and the particle.
Because of this, the scattering is highly forward-peaked. Thus, the magnitude 
of the angular moments of the particle flux distribution decreases very slowly 
and, a high number of moments is necessary to solve the problem accurately.\\
If the scattering ratio $\(c=\frac{\Sigma_{s,0}}{\Sigma_t}\)$ is high, it can be difficult to solve the transport
equation. To speed up the convergence, acceleration schemes such as Diffusion
Synthetic Acceleration (DSA) must be used. DSA schemes focus on accelerating
the weakly anisotropic 
part of the flux, only the zeroth or the zeroth and the first flux 
moments are accelerated \cite{adams}. This is very efficient for neutron transport 
but not for charged-particle transport \cite{multigrid_1d}. The spectral radius of the Source
Iteration (SI) method accelerated by DSA becomes arbitrary close to one when the 
anisotropy increases and the scattering ratio is close to one. Therefore, other methods have been 
developed to accelerate higher order flux moments 
\cite{multigrid_1d,multigrid_2d}.\\
Morel and Manteuffel developed in \cite{multigrid_1d} an angular multigrid
method for the $S_n$ equations in one dimensional geometry. Their angular
multigrid method uses a sequence of lower order transport operator and a
$P_1$ operator to accelerate higher moments. For the scheme to be able to
work, an ``optimal'' transport correction has to be applied to each transport
operator. This transport correction minimizes the ``high-frequency'' angular
errors. The results were very encouraging and Pautz et al. generalized this method 
to multidimensional geometries \cite{multigrid_2d}. This multidimensional
extension of the original angular multigrid method needs to be stabilized
because the scheme amplifies the high-frequency error modes
instead of damping them. They proposed a 
filter which allows the scheme to be always stable but which also reduces the 
efficiency of the method compared to the one dimensional case. However, the spectral 
radius can now become arbitrary close to one when the anisotropy increases.
The angular multigrid methods proposed so far were only applied in the context
of SI but the system of linear equations can also be tackled using non-stationary 
Krylov solvers, such as GMRES. In \cite{ttg}, the authors applied GMRES to
the single group $S_n$ equations with isotropic scattering and 
summarizes the advantageous features of GMRES as \hbox{follows :} ``using DSA as a 
preconditioner for GMRES(m) removes the consistency requirement that plagues 
DSA-accelerated source iteration in multidimensional problems.'' Driven by this
property, we propose to use the angular multigrid as a preconditioner for a 
Krylov solver instead of an acceleration scheme for the Source Iteration
method with application to problems with highly forward-peaked scattering. 
The Krylov solver does not require the method to have a spectral
radius below one to be effective and, therefore, the filter used in
\cite{multigrid_2d} becomes
unnecessary. A code solving the $S_n$ equations using SI with DSA can easily
be modified to use preconditioned Krylov solver. The paper is organized as 
follows; in Section 2, we present the
notation and the discretized transport equation and we recall the
previous works done on this topic and the method that we propose. In Section
3, we show some results and in Section 4, we end with some conclusions.
